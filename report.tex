% //////////////////// Preambolo /////////////////

% Definizione del documento
\documentclass[12pt,a4paper,oneside,hidelinks]{report}

% Lingue usate nel documento e dizionario per la correzione delle parole
\usepackage[english,italian]{babel}

%Codifica dei font di input e di output
\usepackage[T1]{fontenc}
\usepackage[utf8]{inputenc}

% Fornisce i comandi per una buona interlinea dei caratteri
\usepackage{setspace}

% Grafica del documento
\usepackage{graphicx}

% Migliore indentazione del testo
\usepackage{indentfirst}

% Gestisce le caption delle immagini
\usepackage{caption}

% Gestiscono la parte matematica del documento
\usepackage{amssymb, amsmath, amsthm}

% Gestisce il codice del documento
\usepackage{listings}

% Gestisce i link
\usepackage{hyperref}

% Formattazione pagina
\usepackage{geometry}

% Colori
\usepackage{xcolor}

% Pacchetti usati per scrivere lo pseudocodice di un algoritmo
\usepackage{algorithm}
\usepackage[noend]{algpseudocode}

%Rimuove la parola "Algorithm #" dallo pseudocodice
%\captionsetup[algorithm]{labelformat=empty}

% Gestione delle multirighe in una tabella
\usepackage{multirow}

% Gestione delle tabelle su più pagine
\usepackage{subcaption}

% //////////////////////////////////////////////////

% Impostare interlinea a 1.5
\renewcommand{\baselinestretch}{1.5}

% Dichiara la funzione argmin non presente di default
\DeclareMathOperator*{\argmin}{argmin} 

% Apertura del pdf con il 100% di zoom
\hypersetup{pdfstartview={XYZ null null 1.00}}

% Impostare i margini della pagina
\geometry{left=2cm,right=2cm, top=2cm, bottom=2cm}

%Impostare font del documento (Latin Modern Roman)
\renewcommand*\rmdefault{lmr}

% //////////////////// DOCUMENTO /////////////////

\begin{document}

% //////////////////// Titolo /////////////////

%Titolo e intestazione
\title{% 
        Progetto d’esame per il corso di Bioinformatica: 
        Predizione della funzione delle proteine con \\
        metodi di Machine Learning}
  
\author{Federico Picetti \\
        Michele Valsesia}

\date{Anno accademico 2017/2018} 

\maketitle

\tableofcontents

% //////////////////// Capitoli /////////////////


\chapter{Introduzione}

\section{Dati}
Le feature di ingresso sono tratte dalla matrice di adiacenza delle proteine di \emph{Drosophila melanogaster}. La matrice esprime una metrica di similarità fra coppie di proteine.
Gli algoritmi di apprendimento automatico utilizzano l'$ i $-esima riga (o $ i $-esima colonna) della matrice come vettore di feature per l'$ i $-esimo esempio.
Si dispone di 3 distinte matrici di annotazioni, un per ogni ontologia della \emph{GO} (Gene Ontology):
\begin{description}
\item[CC]Cellular Component, 235 classi
\item[BP]Biological Process, 1951 classi
\item[MF]Molecular Function, 234 classi
\end{description}
Si tratta di ontologie multiclasse, per cui ogni proteina può appartenere a una o più classi nella stessa ontologia.
Le matrici di annotazioni $ Y $ riportano le proteine sulle righe e le classi sulle colonne. 

\[ Y_{i,j} =
  \begin{cases}
    1       & \quad \text{se l'elemento } i \text{-esimo appartiene alla classe } j \text{-esima}\\
    0  & \quad \text{altrimenti}\\
  \end{cases}
\]




\chapter{Metodi di machine learning}
Si è deciso di utilizzare tre diversi metodi di Machine Learning: per Support Vector Machine a AdaBoost si sono utilizzate le librerie scikit-learn per Python.

\section{Support Vector Machine}
Si sono provate le SVM della libreria scikit-learn, in particolare l'implementazione SVC\footnote{http://scikit-learn.org/stable/modules/generated/sklearn.svm.SVC.html}.
% Parlare delle impostazioni utilizzate, della funzione di decisione, del bilanciamento, del kernel, di C
\section{AdaBoost}

\chapter{Implementazione}
Eventualmente parlare di dettagli su come è costruito il codice
\chapter{Risultati}
\section{Metriche adottate}
Si è deciso di non utilizzare l'accuratezza in quanto non produce buoni risultati se le classi sono sbilanciate.  
Si è adottata la tecnica di cross-validazione 5-fold. Per ogni classe vengono costruiti 5 classificatori simili e addestrati su 4/5 dei dati. Ogni classificatore viene poi testato sul restante 1/5 dei dati.
L'operazione viene eseguita all'interno del modulo \texttt{metrics.py}, 
%che a sua volta sfrutta la funzione \texttt{cross_val_score} contenuta in 
%\texttt{sklearn.model_selection} 
%\footnote{http://scikit-learn.org/stable/modules/generated/sklearn.model_selection.cross_val_score.html}.
\section{Analisi dei risultati}
\end{document}