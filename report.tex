% //////////////////// Preambolo /////////////////

% Definizione del documento
\documentclass[12pt,a4paper,oneside,hidelinks]{report}

% Lingue usate nel documento e dizionario per la correzione delle parole
\usepackage[english,italian]{babel}

%Codifica dei font di input e di output
\usepackage[T1]{fontenc}
\usepackage[utf8]{inputenc}

% Fornisce i comandi per una buona interlinea dei caratteri
\usepackage{setspace}

% Grafica del documento
\usepackage{graphicx}

% Migliore indentazione del testo
\usepackage{indentfirst}

% Gestisce le caption delle immagini
\usepackage{caption}

% Gestiscono la parte matematica del documento
\usepackage{amssymb, amsmath, amsthm}

% Gestisce il codice del documento
\usepackage{listings}

% Gestisce i link
\usepackage{hyperref}

% Formattazione pagina
\usepackage{geometry}

% Colori
\usepackage{xcolor}

% Pacchetti usati per scrivere lo pseudocodice di un algoritmo
\usepackage{algorithm}
\usepackage[noend]{algpseudocode}

%Rimuove la parola "Algorithm #" dallo pseudocodice
%\captionsetup[algorithm]{labelformat=empty}

% Gestione delle multirighe in una tabella
\usepackage{multirow}

% Gestione delle tabelle su più pagine
\usepackage{subcaption}

% Consente di impostare le virgolette del discorso diretto
\usepackage{dirtytalk}

% //////////////////////////////////////////////////

% Impostare interlinea a 1.5
\renewcommand{\baselinestretch}{1.5}

% Dichiara la funzione argmin non presente di default
\DeclareMathOperator*{\argmin}{argmin}

% Dichiara la funzione sgn non presente di default 
\DeclareMathOperator*{\sgn}{sgn}

% Apertura del pdf con il 100% di zoom
\hypersetup{pdfstartview={XYZ null null 1.00}}

% Impostare i margini della pagina
\geometry{left=2cm,right=2cm, top=2cm, bottom=2cm}

%Impostare font del documento (Latin Modern Roman)
\renewcommand*\rmdefault{lmr}

% //////////////////// DOCUMENTO /////////////////

\begin{document}

% //////////////////// Titolo /////////////////

%Titolo e intestazione
\title{% 
        Predizione della funzione delle proteine \\
        con metodi di Machine Learning}
  
\author{Federico Picetti \\
        Michele Valsesia}

\date{Anno accademico 2017/2018} 

\maketitle

\tableofcontents

% //////////////////// Capitoli /////////////////

\newpage

\section*{Introduzione}
L'obiettivo del progetto consiste nel predire la funzione delle proteine del \textit{Drosophila melanogaster}, un moscerino della frutta, nonché organismo modello per gli insetti, usando dei classificatori prodotti da determinati algoritmi di apprendimento. 

Per affrontare il problema, si è puntato su modelli semplici, rapidi, che consentano di ottenere una buona valutazione dell'errore di classificazione. Gli algoritmi di apprendimento scelti sono: 

\begin{itemize}
    \item Support Vector Machine (SVM)
    \item AdaBoost
\end{itemize}

Ognuno dei metodi sopra elencati verrà applicato alla predizione dei termini MF (Molecular Function) e CC (Cellular Component) della GO (Gene Ontology).

\paragraph*{}
Il progetto è stato svolto utilizzando il linguaggio \textit{Python}, più precisamente la versione 3.5, e la versione 0.18.1 della libreria per l'apprendimento automatico \textit{scikit-learn}. Il caricamento e l'elaborazione delle strutture dati viene svolto usando la versione 0.19.1 della libreria \textit{SciPy}.

\paragraph*{}
L'elaborato descrive i passaggi e le problematiche affrontate durante lo svolgimento del lavoro. Per fare ciò, si è deciso di associare ad ogni singolo stadio lavorativo un capitolo. 

Il \autoref{chap:dati} analizza i dati di input, mostrandone la struttura e le possibili modalità di elaborazione.

Il \autoref{chap:metodi} tratta i metodi di machine learning scelti e la loro implementazione. 

Il \autoref{chap:risultati}, l'ultimo, presenta i risultati ottenuti dai clasificatori, per mezzo di grafici e tabelle, e confronta tra loro gli algoritmi per individuare quello che commette il più basso errore di classificiazione.

\chapter{Analisi dei Dati} 
\label{chap:dati}
Le feature degli esempi da fornire in input agli algoritmi di apprendimento sono contenute in una matrice bidimensionale chiamata \textit{Matrice di Adiacenza}, mentre le rispettive etichette si trovano nella \textit{Matrice delle Annotazioni}.

Per ognuna delle ontologie considerate, il numero delle classi è variabile e quindi si hanno tante \textit{Matrici delle Annotazioni} quante sono le ontologie.

\paragraph*{}
La \textit{Matrice di Adiacenza} è una matrice quadrata simmetrica ed ha una dimensione effettiva di $3195x3195$. Ogni riga identifica una proteina mentre le rispettive colonne rappresentano le feature della proteina stessa. Le entry della matrice contengono una misura di similarità, ottenuta a partire da fattori biologici, fra coppie di proteine.
Ciascuna entry può assumere valori compresi nell'intervallo $[0,1]$. Un valore vicino o uguale a 0 indica che le proteine non sono simili tra loro, mentre valori tendenti o uguali a 1 ne sanciscono la similarità. 
La matrice risulta essere sparsa, a causa dell'elevato numero di entry pari a 0.

\paragraph*{}
La \textit{Matrice delle Annotazioni} è una matrice rettangolare con un numero di righe pari al numero delle proteine e tante colonne quante sono le classi dell'ontologia considerata.

Il numero delle classi per ciascuna ontologia è descritto di seguito:

\begin{itemize}
    \item Cellular Component (CC) si compone di 235 classi
    \item Molecular Function (MF) si compone di 234 classi
\end{itemize}
  
Per ridurre lo spazio in memoria centrale e garantire una rapida elaborazione dei dati, la Matrice delle Adiacenze e la Matrice delle Annotazioni vengono entrambe caricate e convertite in matrici sparse. La funzione utilizzata per la conversione è contenuta nella libreria SciPy.

Le singole colonne delle matrici delle Annotazioni verranno successivamente convertite in array densi, in modo da poterle passare in input ai diversi algoritmi di apprendimento. 

Le classi delle Matrici delle Annotazioni vengono trattate come se fossero indipendenti tra loro, non tenendo conto della gerarchia presente. I classificatori scelti verranno addestrati sulle singole classi e durante la fase di valutazione, si terrà conto di come le loro prestazioni cambiano al variare della classe considerata. 

\chapter{Metodi di Machine Learning} 
\label{chap:metodi}

I metodi di Machine Learning scelti si basano su modelli semplici, rapidi, che consentono di ottenere una buona valutazione dell'errore di classificazione. Gli algoritmi di apprendimento presi in considerazione sono le \textit{Support Vector Machine (SVM)} e \textit{AdaBoost}. La scelta dei metodi è stata effettuata con lo scopo di individuare le differenze tra i modelli e decretarne il migliore. Un'ulteriore motivazione è data dal voler confrontare algoritmi di apprendimento lineari, come SVM, con algoritmi basati sull'utilizzo di più classificatori, come AdaBoost.

\paragraph*{}
Per poter effettuare il tuning degli iperparametri, sono state create diverse configuarazioni di esecuzione.
Le configurazioni sono inserite in un dizionario Python che ha come chiavi i nomi delle configurazioni stesse e come oggetti i classificatori della liberia scikit-learn impostati con i parametri che si vogliono testare.
Utilizzando tutte le classi di un ontologia si ottiene un elevato tempo computazionale per ciascuna configurazione. Per evitare ciò, si è deciso di campionare le classi, prendendone una ogni cinque.
 

\section{Support Vector Machine}
Una \textit{Support Vector Machine}, comunemente abbreviata in \textit{SVM}, costruisce un iperpiano o una serie di iperpiani in uno spazio iperdimensionale i quali possono essere utilizzati per risolvere problemi di classificazione e di regressione. Una buona separazione dei dati si verifica quando si ottiene un iperpiano che massimizza la distanza dal più vicino esempio di ogni classe, in altro modo, quando si individua l'iperpiano di margine massimo. In generale, più è ampio è il margine, più l'errore di classificazione sarà basso.

SVM ricava l'iperpiano di margine massimo risolvendo il seguente problema di ottimizzazione lineare convessa

\begin{equation} \label{uno}
\begin{split}
\min_ {w, b, \zeta} \frac{1}{2} w^T w + C \sum_{i=1}^{n} \zeta_i \\
\textrm {subject to } & y_i (w^T \phi (x_i) + b) \geq 1 - \zeta_i, \\
& \zeta_i \geq 0, i=1, \dotsc ,n
\end{split}
\end{equation}

La sua funzione di decisione è definita come:

\begin{equation} \label{due}
\sgn(\sum_{i=1}^n y_i \alpha_i K(x_i, x) + \rho)
\end{equation}

I support vectors sono gli esempi del training set che, a seconda dell'iperpiano ottenuto, hanno un margine inferiore o pari a 1. La soluzione di SVM dipende solo da questi esempi.

\paragraph*{}
L'apprendimento del classificatore è stato effettuato usando la funzione \textit{SVC} \footnote{scikit-learn.org/stable/modules/generated/sklearn.svm.SVC.html}, implementata nel modulo \textit{sklearn.svm}, contenente gli algoritmi SVM, della libreria scikit-learn. Sono stati considerati i seguenti parametri principali nella definizione del metodo:

\paragraph*{}
\textbf{\textit{C}}. Penalità sull'errore commesso. È un parametro di tradeoff, un valore elevato di C crea un modello complesso che mira a classificare nel miglior modo possibile gli esempi del training set, mentre un valore piccolo comporta un modello con una funzione di decisione più semplice. È un valore di tipo float. 

\paragraph*{}
\textbf{\textit{Kernel}}. Specifica il kernel che deve essere usato dall'algoritmo. I kernel scelti per lo svolgimento del progetto sono stati:

\begin{itemize}
    \item Radial Basis Function (rbf) 
    \item Polinomiale
\end{itemize}

\paragraph*{}
\textbf{\textit{degree}}. Grado del polinomio usato dal Kernel Polinomiale. È un valore intero.

\paragraph*{}
\textbf{\textit{gamma}}. Inverso del raggio di influenza. Questo parametro stabilisce quali sono gli esempi scelti dal modello come support vectors. SVM è molto sensibile a questo parametro. Con un gamma molto grande, il raggio di influenza risulterà piccolo ed il modello sceglierà come support vectors gli elementi più vicini al margine e quelli che violano il vincolo, al contrario, un gamma piccolo comporta un raggio di influenza grande ed ogni esempio del training set sarà un support vector. È un valore di tipo float.

Un valore grande di gamma può portare a overfitting, mentre uno troppo piccolo rischia di produrre un classificatore che non riesce a discriminare al meglio le classi.

\paragraph*{}
\textbf{\textit{class\_weight}}. Parametro utilizzato per il bilanciamento di classi sbilanciate. Se non viene fornito in input, tutte le classi hanno peso unitario. Il peso di ciascuna classe può essere tarato, a seconda del risultato che si vuole ottenere, privilegiando una classe rispetto ad un'altra, oppure calcolato in maniera autonoma per mezzo della modalità \say{balanced}. 
Questa modalità sfrutta i valori delle etichette di una classe per aggiustare i pesi in maniera tale che risultino inversamente proporzionali alle frequenze delle classi di input. In pratica, una classe con una cardinalità molto bassa, avrà un peso alto, al contrario, una classe con cardinalità elevata avrà un peso più basso. La regola di assegnamento dei pesi è la seguente.

\begin{center}
numero\_campioni / (numero\_classi * array\_contenente\_cardinalità\_classi)
\end{center}

\subsection{Implementazione}
Per determinare con quali parametri, sul dataset considerato, la SVM produce dei buoni risultati, vengono eseguite le seguenti configurazioni:

\paragraph*{}
\textbf{\textit{Unbalanced}}. Il kernel di questa configurazione è rbf. La funzione SVC viene eseguita con i parametri di default, C unitario e gamma variabile, per verificare, dall'analisi dei risultati ottenuti, la necessità di modificare gli iperparametri.

\paragraph*{}
\textbf{\textit{C crescente}}. Il kernel usato dalla configurazione è rbf. Il valore di class\_weight è \say{balanced}. Mantenendo costanti tutti gli altri parametri, C viene fatto crescere nell'intervallo $[1,3]$ in maniera discreta. Si vuole verificare che aumentando C in maniera gradata, i risultati subiscano un miglioramento.

\paragraph*{}
\textbf{\textit{C e gamma crescenti}}. Questa configurazione è identica a quella precedente, con la sola differenza di far crescere anche il parametro gamma insieme a C. I valori di entrambi vengono estratti dall'intervallo $[10^(-3), 10^3]$ in scala logaritmica.

\paragraph*{}
\textbf{\textit{Poly}}. Viene usato un kernel polinomiale di grado 4. Il valore di class\_weight è \say{balanced}. Tutti i restanti parametri non hanno subito modifiche. La scelta del grado del polinomio è stata effettuata tenendo in considerazione la potenza computazionale delle macchine a disposizione.

\paragraph*{}
La complessità in tempo di ciascuna configurazione è un $\mathcal{O}(N^2)$ rispetto al numero degli esempi del training set. La velocità di computazione diminuisce considerevolmente all'aumentare della grandezza dei parametri usati.

\section{AdaBoost}
AdaBoost (adaptive boosting) è un algoritmo incrementale che costruisce una serie di classificatori $ h_{i}:\mathbb{R}^{d}\rightarrow \{-1,+1\} $ appartenenti ad una famiglia $ \mathcal{H} $. 
Il procedimento prevede di addestrare un classificatore di base sul training set, calcolarne l'errore, creare copie del classificatore e addestrarle sullo stesso training set, bilanciando i pesi degli elementi del training set classificati scorrettamente dai classificatori precedenti. I classificatori successivi tenderanno quindi a concentrarsi sui casi più difficili.
Al termine avremo un classificatore nella forma
\[\hat{y}=\sgn(\sum_{i=1}^{T} w_{i}h_{i}(\vec{x}))\]
dove $ \vec{w} $ è un vettore di coefficienti reali (pesi) e $ T $ è il numero di classificatori.

Tipicamente contenere i costi computazionali si sceglie una famiglia $ \mathcal{H} $ di classificatori di base molto semplici. $ T $ può essere un numero fissato o può crescere durante l'apprendimento e fermarsi secondo un dato criterio di stop.

In questo lavoro si è usata l'implementazione \texttt{sklearn.ensemble.AdaBoostClassifier}
\footnote{scikit-learn.org/stable/modules/generated/sklearn.ensemble.AdaBoostClassifier.html}.
Questo modulo consente di specificare i seguenti parametri:

\begin{description}
\item[\texttt{base\_estimator}:]la famiglia $ \mathcal{H} $ di \emph{weak learner};
\item[\texttt{n\_estimators}:]la quantità massima di stimatori da costruire, limite oltre il quale il boosting è terminato.
In caso di aderenza perfetta, la procedura di boosting è arrestata prima del raggiungimento del limite.
\item[\texttt{learning\_rate}:] permette di ridurre il contributo di ogni classificatore.
\item[\texttt{algorithm}:]consente di scegliere fra due algoritmi:
\begin{description}
\item[SAMME]Algoritmo di boosting discreto: necessario quando $ \mathcal{H} $ è una famiglia di decisori;
\item[SAMME.R]Algoritmo di boosting reale: tipicamente converge più velocemente, praticabile quando i classificatori $ \mathcal{H} $ restituiscono una probabilità di appartenenza per ogni classe.
\end{description}
\end{description}
 
\subsection{Parametri}
In questo lavoro sono state esplorate diverse configurazioni del predittore AdaBoost.
Come classificatore di base si è optato per un \emph{decision stump}, cioè un predittore ad albero molto semplice, che presenta un solo test, quindi una sola ramificazione e al massimo due foglie. Si è utilizzata l'implementazione dell'albero di decisione di scikit-learn\footnote{scikit-learn.org/stable/modules/generated/sklearn.tree.DecisionTreeClassifier.html}, impostandone il parametro \texttt{max\_depth} a 1.

Si è variato \texttt{n\_estimators} ponendolo a 10, 50 e 100 stimatori di base.

Si è poi andato a modificare alcune proprietà dello stimatore ad albero di base, impostando un bilanciamento del peso delle classi inversamente proporzionale alla loro frequenza.
È stato fatto un tentativo utilizzando alberi leggermente più complessi, impostandone la \texttt{max\_depth} a 3.

\begin{table}[ht]%	
\centering
\caption{Cosa fa la tabella?}\label{tab:b1}
\begin{tabular}{|c|c|c|c|}
\hline
nome & \texttt{n\_estimators} & bilanciamento & \texttt{max\_depth} \\ 
\hline 
AdaBoost\_n10 & 10 & No & 1 \\ 
\hline 
AdaBoostDefault & 50 & No & 1 \\ 
\hline 
AdaBoost\_n100 & 100 & No & 1 \\ 
\hline 
AdaBoost\_n5\_Bal & 5 & Auto & 1 \\ 
\hline 
AdaBoost\_n10\_Bal & 10 & Auto & 1 \\ 
\hline 
AdaBoost\_n50\_Bal & 50 & Auto & 1 \\ 
\hline 
AdaBoost\_n50\_Bal\_Dep3 & 5 & Auto &  3\\ 
\hline 
\end{tabular} 
\end{table}

\subsection{Implementazione}
Eventualmente parlare di dettagli su come è costruito il codice

\section{Costruzione del software}
Il programma Python è costruito in modo tale da calcolare le funzioni di predizione per un'unica ontologia. Questa scelta è stata effettuta per facilitare la gestione del codice e dei risultati prodotti. Se si vogliono eseguire diverse ontologie bisogna mandare nuovamente in esecuzione il processo specificando l'ontologia che si vuole computare.
Tuttavia, per una singola istanza del programma, possono essere inserite più configurazioni contemporaneamente.

Gli step seguiti dall'algoritmo sono i seguenti:

\begin{itemize}
    \item Verificare che l'ontologia e le configurazioni inserite siano corrette
    \item Se ritenute valide, le configurazioni vengono eseguite una alla volta sulle \\     
    classi dell'ontologia
    \item Calcolo delle metriche e salvataggio dei risultati ottenuti
\end{itemize}

\paragraph*{}
Le performance di un metodo vengono valutate con la tecnica sperimentale della 5-fold cross-validation. Questa tecnica suddivide il dataset in 5 fold, quattro dei quali vengono usati per addestrare il classificatore e il fold restante per ottenere le metriche richieste. Ogni fold deve essere utilizzato almeno una volta come test set, questo implica che per ogni classe vengono prodotti cinque classificatori, ciascuno con le proprie metriche.

I risultati ottenuti dall'esecuzione di una configurazione sono salvati in un file \textit{json} che presenta la seguente struttura:

\begin{itemize}
    \item Header
    \item Metriche dei fold di una classe
    \item Footer
\end{itemize}
L'\textit{Header} è un dizionario contenente le seguenti informazioni:
\begin{itemize}
    \item L'ontologia in uso
    \item L'algoritmo di apprendimento eseguito
    \item I parametri del metodo
    \item Il tempo di inizio, in formato data e ora, della configurazione
\end{itemize}
Le \textit{Metriche} ricavate da ciascun fold di una classe sono le seguenti:
\begin{itemize}
\item Identificativo della classe (indice colonna della Matrice delle Annotazioni)
\item Numero di esempi positivi nel fold
\item Precision
\item Recall
\item F-score
\item False Positive rate
\item True Positive rate
\item Auroc
\item Auprc
\end{itemize}
Anche la struttura sopra descritta è contenuta all'interno di un dizionario. Ad ogni classe vengono associati tanti dizionari quanti sono i fold.

Il \textit{Footer} è costituito da un dizionario di un solo elemento: il tempo di fine, in formato data e ora, della configurazione.

Dai file json verranno estratte le informazioni necessarie per la valutazione dei classificatori.

\paragraph*{}
Per poter eseguire la cross-validation e calcolare le misure richieste, vengono utilizzate delle funzioni messe a disposizione dalla libreria scikit-learn.
La funzione \textit{cross\_val} è stata scelta perché riduce il tempo di computazione, in quanto assegna l'esecuzione dei differenti fold di una classe alle diverse unità di elaborazione presenti su una determinata macchina.

Le metriche dei fold vengono calcolate a partire dalle etichette degli esempi contenuti nel test set e dalle predizioni del classificatore del fold. Di seguito una breve lista delle funzioni utilizzate:

\begin{itemize}
\item \textit{precision\_recall\_fscore\_support} calcola la precision, la recall e la f-score di un classificatore

\item \textit{precision\_recall\_curve} computa la precision-recall-curve

\item \textit{average\_precision\_score} restituisce l'AUPRC del classificatore 

\item \textit{roc\_curve} determina la ROC

\item \textit{auc} restituisce l'AUROC del classificatore
\end{itemize}


\chapter{Analisi dei Risultati}
\label{chap:risultati}

\section{Scelta delle metriche}
Si è deciso di non utilizzare l'accuratezza in quanto non produce buoni risultati se le classi sono sbilanciate.  
Si è adottata la tecnica di cross-validazione 5-fold. Per ogni classe vengono costruiti 5 classificatori simili e addestrati su 4/5 dei dati. Ogni classificatore viene poi testato sul restante 1/5 dei dati.
L'operazione viene eseguita all'interno del modulo \texttt{metrics.py}, 
che a sua volta utilizza la funzione \texttt{cross\_val\_score} contenuta in 
\texttt{sklearn.model\_selection} 
\footnote{scikit-learn.org/stable/modules/generated/sklearn.model\_selection.cross\_val\_score.html}. 

\section{Risultati}
Scrivi qui Pici come devono essere interpretati i risultati, la spiegazione dei grafici, quello che volevamo confrontare e il significato del risultato ottenuto.

Da qui in poi crei le pagine per i grafici 

\chapter{Conclusioni}

\end{document}