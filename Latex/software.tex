\section{Costruzione del software}
Il software realizzato in Python calcola le funzioni di predizione per tutte le classi dell'ontologia. Inoltre, per ogni singola istanza del programma, possono essere eseguite più configurazioni contemporaneamente.

Gli step seguiti dall'algoritmo sono i seguenti:

\begin{itemize}
    \item Verificare che le configurazioni inserite siano corrette
    \item Se ritenute valide, le configurazioni vengono eseguite una alla volta sulle classi dell'ontologia
    \item Calcolo delle metriche e salvataggio dei risultati ottenuti
\end{itemize}

\paragraph*{}
Le performance di un metodo vengono valutate con la tecnica sperimentale della 5-fold cross-validation. Questa tecnica suddivide il dataset in 5 fold, quattro dei quali vengono usati per addestrare il classificatore, mentre il quinto corrisponde al test set e viene usato per il calcolo delle metriche richieste. Ogni fold deve essere utilizzato almeno una volta come test set, questo implica che per ogni classe vengono prodotti cinque classificatori, ciascuno con le proprie metriche.

I risultati ottenuti dall'esecuzione di una configurazione sono salvati in un file \textit{json} che presenta la seguente struttura:

\begin{itemize}
    \item Header
    \item Metriche dei fold di ogni classe
    \item Footer
\end{itemize}
L'\textit{Header} è un dizionario contenente le seguenti informazioni:
\begin{itemize}
    \item L'ontologia in uso
    \item L'algoritmo di apprendimento eseguito
    \item Il tempo di inizio della configurazione in formato data e ora
\end{itemize}
Le \textit{Metriche} ricavate da ciascun fold di ogni classe sono le seguenti:
\begin{itemize}
    \item Identificativo della classe (indice colonna della Matrice delle Annotazioni)
    \item Numero di esempi positivi nel fold
    \item Precision
    \item Recall
    \item F-score
    \item False Positive rate
    \item True Positive rate
    \item AUROC
    \item AUPRC
\end{itemize}

Una volta caricato in memoria, il file \textit{json} viene salvato come dizionario. Ad ogni classe sono associati tanti dizionari quanti sono i fold.

Il \textit{Footer} è costituito da un dizionario di un solo elemento: il tempo di fine della configurazione in formato data e ora.

Dai file \textit{json} prodotti vengono estratte le informazioni necessarie per la valutazione dei classificatori.

\paragraph*{}
Per poter eseguire la cross-validation e calcolare le misure richieste, vengono utilizzate delle funzioni messe a disposizione dalla libreria scikit-learn.
La funzione \textit{cross\_val\_score} è stata scelta perché consente di ridurre il tempo di computazione, assegnando l'esecuzione dei differenti fold di una classe alle diverse unità di elaborazione presenti sulla macchina.

Le metriche di un fold vengono calcolate dando in input alle funzioni le etichette degli esempi del test set e le predizioni computate dal classificatore. Di seguito una breve lista delle procedure utilizzate:

\begin{itemize}
\item \textit{precision\_recall\_fscore\_support} calcola la Precision, la Recall e la F-score di un classificatore

%\item \textit{precision\_recall\_curve} computa la precision-recall-curve

\item \textit{average\_precision\_score} restituisce l'AUPRC del classificatore 

%\item \textit{roc\_curve} determina la ROC

\item \textit{auc} restituisce l'AUROC del classificatore
\end{itemize}