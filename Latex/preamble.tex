% //////////////////// Preambolo /////////////////

% Definizione del documento
\documentclass[12pt,a4paper,oneside,hidelinks]{report}

% Lingue usate nel documento e dizionario per la correzione delle parole
\usepackage[english,italian]{babel}

%Codifica dei font di input e di output
\usepackage[T1]{fontenc}
\usepackage[utf8]{inputenc}

% Fornisce i comandi per una buona interlinea dei caratteri
\usepackage{setspace}

% Grafica del documento
\usepackage{graphicx}

% Migliore indentazione del testo
\usepackage{indentfirst}

% Gestisce le caption delle immagini
\usepackage{caption}

% Gestiscono la parte matematica del documento
\usepackage{amssymb, amsmath, amsthm}

% Gestisce il codice del documento
\usepackage{listings}
\renewcommand{\ttdefault}{txtt}

% Gestisce i link
\usepackage{hyperref}

% Formattazione pagina
\usepackage{geometry}

% Colori
\usepackage{xcolor}

% Pacchetti usati per scrivere lo pseudocodice di un algoritmo
\usepackage{algorithm}
\usepackage[noend]{algpseudocode}

%Rimuove la parola "Algorithm #" dallo pseudocodice
%\captionsetup[algorithm]{labelformat=empty}

% Gestione delle multirighe in una tabella
\usepackage{multirow}

% Gestione delle tabelle su più pagine
\usepackage{subcaption}

% Consente di impostare le virgolette del discorso diretto
\usepackage{dirtytalk}

% Allineamento dei numeri in una tabella
\usepackage{siunitx}

% Converte file eps in pdf
\usepackage{epstopdf}

% Commenta ed esclude parti del file latex dalla compilazione
\usepackage{comment}

% L'output è un pdf a-1b
%\usepackage[a-1b]{pdfx} 

% //////////////////////////////////////////////////

% Impostare interlinea a 1.5
\renewcommand{\baselinestretch}{1.5}

% Dichiara la funzione argmin non presente di default
\DeclareMathOperator*{\argmin}{argmin}

% Dichiara la funzione sgn non presente di default 
\DeclareMathOperator*{\sgn}{sgn}

% Apertura del pdf con il 100% di zoom
\hypersetup{pdfstartview={XYZ null null 1.00}}

% Impostare i margini della pagina
\geometry{left=2cm,right=2cm, top=2cm, bottom=2cm}

% Reinclude nella compilazione le parti escluse
%\includecomment{comment}

%Impostare font del documento (Latin Modern Roman)
\renewcommand*\rmdefault{lmr}

% Definizione dei colori per i commenti, le keyword e le stringhe dei codici
\definecolor{bluekeywords}{rgb}{0.13,0.13,1}
\definecolor{greencomments}{rgb}{0,0.5,0}
\definecolor{redstrings}{rgb}{0.9,0,0}

% Opzioni grafiche delle liste contenenti il codice scritto in Python
\lstdefinestyle{customp}{
    language=Python, 
    basicstyle=\ttfamily\footnotesize,   
    backgroundcolor=\color{gray!10},
    frame=none,
    tabsize=2,
    commentstyle=\color{greencomments},
    keywordstyle=\color{bluekeywords},
    stringstyle=\color{redstrings},
    title=\lstname,    
    escapeinside={\%*}{*)},
    breaklines=true,
    breakatwhitespace=true,    
    framextopmargin=2pt,
    framexbottommargin=2pt,
    inputencoding=utf8,
    extendedchars=true,
    showstringspaces=false,
    literate={à}{{\'a}}1 {ã}{{\~a}}1 {é}{{\'e}}1 {ù}{{\'u}}1,
}